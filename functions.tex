% Preamble
% --------
%
% amsmath : American Mathematical Society
% amssymb : American Mathematical Society symbols
% bm : Bold symbols in Math mode

\documentclass{article}
\usepackage{amsmath}
\usepackage{amssymb}
\usepackage{bm}
\usepackage{graphicx}
\usepackage{pgfplots}
\usepackage{gensymb}
\usepackage{blindtext}
\usepackage{titlesec}
\usepackage{tikz}
\usepackage{tikz-3dplot}
\pgfplotsset{compat=1.18}

\title{Notes about various mathematical topics.}
\author{Craig Sanders}
\date{ }


\begin{document}

\newpage

\maketitle

\newpage

\tableofcontents

\newpage

\listoffigures

% Document body

\newpage

\section{Introducting mathematical functions.}

\subsection{Numbers and number lines.}

Think of all the individual numbers which we might possibly use throughout the course of our everyday lives, $0$, $1$, $2$, 
$3$, $10$, $-100$, $3.141$, $0.000001$, $3817/245$ and so on. All of these numbers are called real numbers and
are denoted by the symbol $\mathbb{R}$.
We can think of every real number as residing on an infinitely long straight line which stretches from
$-\infty$ to $+\infty$. We call this infinitely long straight line the real number line, and there is a place on this
line for any and all real numbers that we can think of.


\subsection{Imaginary numbers.}

There is however, a problem with this real number line and that is that it doesn't contain all the numbers!\\ 

Consider the number $-1$ and what will happen when we take the square root of it. The result will be $i$, where 
$i$ stands for ``imaginary number'' and unfortunately there is no place for this number -- or any multiples of it,
i.e. $i2$, $i100$, $-i707.87$ and so on, on the real number line. There is however, a place for $i$ on another
number line which we call the imaginary number line. We can also think
of this line as an infinitely long straight line, but instead of stretching from $-\infty$ to $+\infty$ like the real
number line, this one stretches from $-i\infty$ to $+i\infty$


\subsection{The weird world of the imaginary number line.}

The imaginary number line is a bit of a weird one. In many ways it is just like the real number line, but in some ways
it is not. If you add or subrtract multiple imaginary numbers together, you get another imaginary number. For example;

\begin{align*}
i + i =& i2 \\
i2 + i3 + i4 =& i7 \\
i50 - i100 =& -i50 
\end{align*}

This behaviour is just like the real number line. However, consider what happens when we start multiplying or dividing
imaginary numbers with each other.

\begin{align*}
i2 \times i3 =& -6 \\
i10 \div i5 =& 2 
\end{align*}

We can see here in these examples, that when an imaginary number is multiplied or divided by another imaginary
number, the result is a real number -- not another imaginary number. This is not what happens with real numbers.
With real numbers, multiplying or dividing one real number by another real number, yields a result that is
also a real number. That is, the types of the operands and the result in this case are all homogenous, i.e. 
they are all real numbers. This is an interesting observation which we will discuss in the next section. 

\subsection{Operation type transitivity.}

In the previous section, we discussed one of the differences between real and imaginary numbers. This difference
occurs when one of these numbers is multiplied with another number of the same type. I've devised a term for this
phenomenon because I can't find a pre-existing term which describes it; I call it ``operation type transivity'', and
it can take on the values of ``homogeneous'', ``heterogeneous'', or ``bigeneous''.\\

If a mathematical operator takes operands which are of a certain type -- say for example real numbers, and it
generates one or more values which are of this same type, then this function's operation type transivity is
referred to as being ``homogeneous''. This is excatly what happens when we multiply two real numbers together.\\

On the other hand, if a mathematical operator takes operands which are of another certain type -- say for example 
imaginary numbers this time, and it generates one or more values which are of a different type, then this function's 
operation type transivity is referred to as being ``heterogeneous''. This is exactly what happens when we multiply two
imaginary numbers together.\\

But what about operations which don't conform to concept of operation type homogeneity? We saw earlier that when 
two imaginary numbers are multiplied or divided with each other, the result is real. In this case, the operation
could be thought of as exhibiting ``operation type heterogeneity''.


\subsection{Multiplication and division of imaginary numbers.}

Why does multiplying or dividing two imaginmary numbers together result in heterogeneous operation types?
To help try and answer this, recall the definition of an imaginary number, that is;

\begin{equation*}
i = \sqrt{-1}
\end{equation*}

Therefore;

\begin{equation*}
i^{2} = -1
\end{equation*}

So for example, in the case where $i2 \times i3$, we essentially have;

\begin{align*}
i2 \times i3 =&  (i \times 2) \times (i \times 3) \\
             =&  i \times 2 \times i \times 3
\end{align*}

Re-arranging this yields;

\begin{align*}
i2 \times i3 =&  (i \times i) \times (2 \times 3) \\
             =&  i^{2} \times 6 \\
             =&  -1 \times 6 \\
             =&  -6
\end{align*}

Similarly, if we consider the division operation from above;

\begin{align*}
i10 \div i5 =& (i \times 10) / (i \times 5)
\end{align*}

Seeing as $i$ is common in both the numerator and the denominator, we can simply cancel them out with each other,
leaving a result of $2$.\\

What happens however, if we multiply or divide three imaginary numbers with each other? Consider the following
examples;

\begin{align*}
i2 \times \i3 \times i4 \\
(i24 \div i4) \div 3 
\end{align*}


\subsubsection{Loss of computing power.}



From the previous section, we can see that imaginary numbers start to lose or take on altered computational
abilities compared to real numbers.


\subsection{Complex numbers.}

Consider the following mathematical expression;

\begin{equation*}
x + y
\end{equation*}

What is the deal with this expression when $x=1$ and $y=i1$? We can see that the result is comprised of two
parts; a real part $1$ and an imaginary part $i1$. Due to the fact that this number is comprised of two parts, it is
referred to as a ``complex number''. You might now be tempted to ask, which number line does this result reside on? The
answer to that question is that it resides on both the real and imaginary number lines. However, perhaps a better
answer is that
the result of this particular instance of the expression resides in a plane -- a complex plane to be more precise.
Let's elaborate a bit more in an attempt to try and explain exactly what this means. If we arrange the real and imaginary number
lines at right angles to each other, such that they intersect at 0, then we have what is known as a complex plane.
This value of $1 + i1$ can now be plotted on this complex plane.


\subsection{Denoting a function.}

Consider a function which is dependent on three separate and independent variables, \begin{math} x\end{math}, $y$ and 
\begin{math} z\end{math}. We can denote this function with the letter \begin{math} f \end{math} as follows;

\begin{equation*}
  f(x, y, z)
\end{equation*}

Note that within the declaration of this function, the three separate variables \begin{math} x\end{math}, 
\begin{math}y\end{math} and \begin{math} z\end{math}, are listed within parentheses.\\

This function could conceivably be used to model something like the colour represented by any point within a 
colour sphere; where \begin{math} x \end{math} could represent the forward and backward position within the
colour sphere, i.e. the yellow and blue axis, \begin{math} y \end{math} could represent the left and right position
within the colour sphere, i.e. the red and green axis, and \begin{math} z \end{math} could represent the up and down
position within the colour sphere, i.e. the black and white axis.

\newpage

\begin{figure}
  \includegraphics[scale=0.75]{images/Color_sphere.png}
  \caption{This figure depicts a 3-dimensional color sphere.}
  \label{fig:Color_sphere}
\end{figure}


\subsection{Defining a function.}

Let us now provide a definition or body for our function \begin{math} f \end{math}. We will see in the next section
why we have chosen this as the definition for our function -- that being, because it raises some interesting questions
about imaginary and complex numbers.

\begin{equation}
  f(x,y,z) = \sqrt{x} + \sqrt{y} + \sqrt{z}
\end{equation}

Note that the definition of the function is comprised of three distinct parts, $\sqrt{x}$, $\sqrt{y}$, and
$\sqrt{z}$, all of which are added together to yield the resulting value for the function. It is worth explicitly
mentioning that each of these three distinct parts is orthogonal to the other two, so they can't all simply be 
added together in such an easy manner as one might otherwise think!\\

Let's now put this function to one side for a moment, while we briefly discuss matrices and vectors, and how they might
help us with our definition of this function.


\subsection{Matrices.}

When presented in printed form, a matrix appears as a 2-dimensional mathematical construct
which is comprised of both rows and columns of values. The dimensions of any particular matrix can be described
as an \textit{m} x \textit{n} matrix -- where \textit{m} denotes the number or rows that make up the matrix, and
\textit{n} denotes the number of columns that make up the matrix. An example of a 2 x 3 
matrix, called $A$ in this case, is shown below;

\begin{equation*}
A = 
\begin{bmatrix}
2 & 8 & 5\\
7 & 4 & 1
\end{bmatrix}
\end{equation*}

where column 1 is comprised of the values 2 and 7, column 2 is comprised of the values 8 and 4, and column 3 is
comprised of the values 5 and 1. Similarly, row 1 is comprised of the values 2, 8, and 5, while row 2 is comprised
of the values 7, 4, and 1. It should be obvious now, that the reason we refer to this as a 2 x 3 matrix, is because
it is composed of 2 rows and 3 columns worth of values.\\

Consider the following function definition;

\begin{equation*}
f(x,y,z) = 2x + 3y + 4z
\end{equation*}

This function can be represented by the multiplication of two matrices as is shown below;

\begin{equation*}
f(x,y,z)= 
\begin{bmatrix}
2 & 3 & 4
\end{bmatrix}
\times
\begin{bmatrix}
x \\
y \\
z
\end{bmatrix}
\end{equation*}

 
\subsection{Vectors.}

Vectors are simply matrices which have a value of 1 for one of their dimensions. A benefit of vectors, is that 
they allow multi-part or multi-dimensional formula to be expressed in a more compact or succint manner. For example,
the function from earlier could be written in vector format as follows;

\begin{equation*}
f(x,y,z) = 
\begin{bmatrix}
\sqrt{x} & \sqrt{y} & \sqrt{z}
\end{bmatrix}
\end{equation*}

or, if a vertically oriented layout is preferred;

\begin{equation*}
f(x,y,z) = 
\begin{bmatrix}
\sqrt{x} \\
\sqrt{y} \\
\sqrt{z}
\end{bmatrix}
\end{equation*}



\subsection{Unit vectors.}

When a particular unit vector is used -- say for example
on a particular number line, then it denotes this number line as being orthogonal to all of the other number lines.
Since $\hat{\mathrm{\bm{i}}}$, $\hat{\mathrm{\bm{j}}}$, and $\hat{\mathrm{\bm{k}}}$ have now been applied
to the three parts of the function definition, we can now add them together using the mathematical addition operation,
i.e. $+$ without causing unnecessary confusion. This means that the three parts will add together in an orthogonal
sense to produce an overall 3-dimensioanl result.\\

Don't be confused by unit vectors. They can be thought in a sense as being a bit like direction readings when bush
walking. However, instead of being called north/south, east/west, or up/down, they are referred to as 
$\hat{\mathrm{\bm{i}}}$, $\hat{\mathrm{\bm{j}}}$, and $\hat{\mathrm{\bm{k}}}$. The image in \ref{fig:3d_axis} depicts what the three unit vectors look like, when 
they are overlaid onto a 3-dimensional Cartesian space which is comprised of x, y, and z axes.

\begin{figure}
  \includegraphics[scale=1.0]{images/3d_axis.jpg}
  \caption{This figure depicts a 3-dimensional Cartesian co-ordinate space $\mathbb{R}^3$ which is comprised
  of three orthogonal axes labelled x, y, and z. Overlaid in blue onto each of these three axes, are the three
  unit vectors, $\hat{\mathrm{\bm{i}}}$, $\hat{\mathrm{\bm{j}}}$, and $\hat{\mathrm{\bm{k}}}$.}
  \label{fig:3d_axis}
\end{figure}


\subsection{Complex vectors.}

Just as vectors can contain real numbers, they can also contain imaginary or complex numbers. For example;

\begin{equation*}
f(x,y,z) = 
\begin{bmatrix}
2 + i7 \\
5 + i3 \\
3 - i9
\end{bmatrix}
\end{equation*}

Could this function definition also be defined as;

\begin{equation*}
f(x,y,z) = \hat{\mathrm{\bm{i}}}(2 + i7) + \hat{\mathrm{\bm{j}}}(5 + i3) + \hat{\mathrm{\bm{k}}}(3 - i9)
\end{equation*}

This definition can potentially be thought of as being comprised of 6 dimensions. That is 3 dimensions in
the real space, i.e. $x$, $y$, and $z$, and a corresponding 3 dimensions in the imaginary domain. So how then, could
one go about plotting this function definition when it is comprised of so many dimensions?

\newpage
\section{Calculating the function.}

Earlier, we stated that we would assign the following definition to our function $f(x,y,z)$;

\begin{equation*}
  f(x,y,z) = \langle \sqrt{x}, \sqrt{y}, \sqrt{z} \rangle
\end{equation*}

Values of this function are simple enough to calculate when $x$, $y$, and $z$ are $\geq 0$, but what about when
these values are less than 0, for example;

\begin{align*}
  f(x,y,z) &= \langle \sqrt{-1}, \sqrt{-1}, \sqrt{-1} \rangle \\
           &= \hat{\mathrm{\bm{i}}}\sqrt{-1} + \hat{\mathrm{\bm{j}}}\sqrt{-1} + \hat{\mathrm{\bm{k}}}\sqrt{-1}
\end{align*}

According to the Wolfram Alpha website;

\begin{quote}
``Quaternions are a 4-dimensional number system that is an
extension of the field of complex numbers. A quaternion can be visualized as a rotation of vectors in three dimensions.''
\end{quote}

\subsection{Visualising the function by plotting it.}

\subsection{Limitations with visualising functions.}

\input{./subsections/square_root_x_y.tex}


\subsection{1-dimensional functions.}

1-dimensional functions don't rely upon any independent variables. For example, consider the following
mathematical expressions;

\begin{align*}
x^{2} =& 1 \\
x     =& 5 \\
y     =& \pi
\end{align*}

Let us take the first example and give it some further consideration. The first example can be re-arranged
as follows;

\begin{align*}
x &= \pm\sqrt{1} \\
  &= \pm1
\end{align*}

Plotting this simply results in two points on the 2-dimensional $x-y$ Argand diagram.

\begin{figure}
  \includegraphics[scale=1.5]{images/Plus_and_minus_one.jpg}
  \caption{This figure depicts the result of plotting $x = \pm1$ on a 2-dimensional Argand diagram.}
  \label{fig:Plus_and_minus_one}
\end{figure}


\subsection{2-dimensional functions.}

2-dimensional functions rely upon a single independent variable. For example, consider the following mathematical
expressions;

\begin{align*}
f(x) =& x^{2} \\
f(x) =& x + 5 \\
y    =& x - \pi
\end{align*}

Note that the final example uses different nomenclature to the other two examples. The final example assigns the
result of the expression, i.e. $x - \pi$, to the dependent variable $y$.

Let us now expand the example which we singled out for further consideration in the previous section. We will
expand and rewrite this expression as follows;

\begin{equation}
x^{2} + y^{2} = 1
\end{equation}

This expression plots what is called a unit circle; that is a circle with radius 1 and which is centred on the origin 
of the $x-y$ plane. We can easily plot this expression on a 2-dimensional $x-y$ Argand diagram as is shown in figure .

\begin{figure}
  \begin{center}
  \includegraphics[scale=1.5]{images/Unit_circle.jpg}
  \end{center}
  \caption{This figure depicts a unit circle on a 2-dimensional Argand diagram.}
  \label{fig:Unit_circle}
\end{figure}


\subsection{3-dimensional functions.}

Let us now expand the example from the previous section, even further again. We will do this as follows;

\begin{equation}
x^{2} + y^{2} + z^{2} = 1
\end{equation}

This expression plots what is called a unit sphere; that is a 3-dimensional sphere with a radius of 1 and
which is centred on the origin of the $x-y-z$ space. An image of such a unit sphere is depicted in
\ref{fig:Unit_sphere}.

% \input{./subsections/sphere.tex}

\begin{figure}
  \begin{center}
  \includegraphics[scale=0.5]{images/Sphere_with_x_y_z_axes.png}
  \end{center}
  \caption{This figure depicts a unit sphere within a 3-dimensional co-oridnate axis system. Note that the $x$, $y$
           and $z$ axes have been coloured red, green and blue respectively. This has been done intentionally in an
           attempt to try and make them stand out against the unit sphere, which has been rendered in monochrome.}
  \label{fig:Unit_sphere}
\end{figure}


\subsection{4-dimensional functions.}

Let us once again expand on the example from the previous section. This time we will do this as follows;

\begin{equation}
x^{2} + y^{2} + z^{2} + a^{2} = 1
\end{equation}


\end{document}